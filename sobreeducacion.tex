\documentclass{article}
\usepackage[utf8]{inputenc}

\begin{document}

La \textit{evaluación formativa} es un proceso que tiene como objetivo mejorar el aprendizaje de los alumnos mediante la retroalimentación continua y el ajuste de las estrategias de enseñanza \cite{fuente4}. No se centra en calificar o clasificar a los alumnos, sino en identificar sus fortalezas y debilidades, así como las oportunidades de mejora. Implica una participación activa de los alumnos, los docentes y los pares, y se basa en el uso de diversos instrumentos y técnicas, como observaciones, cuestionarios, autoevaluaciones, coevaluaciones, portafolios, etc. \cite{fuente4}.

La \textit{certificación de conocimiento} es un proceso que tiene como objetivo verificar el nivel de competencia de los alumnos en un determinado campo o disciplina mediante la aplicación de pruebas o exámenes estandarizados y validados \cite{fuente2}. Se centra en medir el rendimiento o los resultados de los alumnos y se utiliza para otorgar títulos, diplomas, acreditaciones o reconocimientos. Implica una evaluación externa e independiente de los programas o instituciones educativas y se basa en el uso de criterios y estándares de calidad, como indicadores, rubricas, matrices, etc. \cite{fuente2}.

La relación entre la evaluación formativa y la certificación de conocimiento en la educación universitaria es compleja y dinámica. Por un lado, la evaluación formativa puede contribuir a mejorar la calidad de la educación y a preparar a los alumnos para enfrentar los desafíos de la certificación de conocimiento. Por otro lado, la certificación de conocimiento puede influir en las prácticas de evaluación formativa y en las expectativas de los alumnos y los docentes. Ambos procesos deben estar alineados con el currículo, los objetivos de aprendizaje y las competencias que se pretenden desarrollar en los alumnos \cite{fuente4}.

Aquí hay algunos artículos relacionados con los temas expuestos:

\textbf{Evaluación formativa en la educación universitaria:}
Este tipo de evaluación se enfoca en el proceso de aprendizaje de los estudiantes y les brinda retroalimentación oportuna y constructiva para mejorar su desempeño. Algunos artículos que abordan este tema son:
\begin{itemize}
    \item \textbf{La evaluación formativa en la educación 2:} Este artículo de revisión analiza la importancia de la evaluación formativa en la educación, sus características, ventajas y desafíos. \cite{fuente2}
    \item \textbf{Evaluación formativa como desafío de la educación universitaria ante la pandemia por COVID-19 3:} Este artículo reflexiona sobre la evaluación formativa en los entornos virtuales de aprendizaje, sus beneficios y estrategias para implementarla. \cite{fuente3}
    \item \textbf{Tendencias de la evaluación formativa y sumativa del alumnado en Web of Science 4:} Este artículo presenta una revisión bibliométrica de las publicaciones sobre evaluación formativa y sumativa en la base de datos Web of Science, identificando los autores, países, revistas e instituciones más relevantes. \cite{fuente4}
\end{itemize}

\textbf{Certificación de conocimiento en la educación universitaria:}
Este tipo de certificación se refiere al reconocimiento oficial de las competencias y habilidades adquiridas por los estudiantes en su formación académica. Algunos artículos que tratan este tema son:
\begin{itemize}
    \item \textbf{La evaluación y acreditación de la educación profesional en México: ¿la calidad educativa en juego? 6:} Este artículo examina el proceso de evaluación y acreditación de los programas de educación superior en México, sus antecedentes, objetivos, actores y resultados. \cite{fuente6}
    \item \textbf{La acreditación del aprendizaje por experiencia en la educación universitaria 7:} Este artículo describe y analiza el proceso de acreditación del aprendizaje por experiencia en el Instituto Pedagógico de Miranda José Manuel Siso Martínez de la Universidad Pedagógica Experimental Libertador, sus fundamentos, fases y criterios. \cite{fuente7}
    \item \textbf{La evaluación y acreditación de la educación superior en América Latina 5:} Este artículo ofrece una visión panorámica de la evaluación y acreditación de la educación superior en América Latina, sus antecedentes, modelos, tendencias y desafíos. \cite{fuente5}
\end{itemize}


El rezago educativo se refiere al fenómeno de que los estudiantes no logran completar sus estudios en el tiempo previsto o abandonan el sistema educativo sin obtener un título. Es un problema que afecta a la calidad y la equidad de la educación, así como al desarrollo económico y social de un país. Algunos de los factores que se asocian al rezago educativo son la selección de carga académica, la falta de cursos esenciales o con seriación, la reprobación, los horarios inapropiados, el nivel socioeconómico, el género, la edad, el estado civil, la situación laboral, el apoyo familiar y la satisfacción con la carrera \cite{fuente7, fuente8}.

Los indicadores para medir el rezago educativo son herramientas estadísticas que permiten cuantificar y comparar el desempeño de los estudiantes, las instituciones y los sistemas educativos. Algunos de los indicadores más utilizados son la tasa de deserción, la tasa de reprobación, la tasa de titulación, el tiempo promedio de egreso y el índice de rezago. Estos indicadores se pueden calcular a nivel de carrera, de institución, de región o de país, y se pueden desagregar por variables como el género, el tipo de institución, el tipo de programa, el turno, el nivel socioeconómico, etc. \cite{fuente7, fuente9}.

La eficiencia terminal se define como la relación entre el número de alumnos que se inscriben por primera vez a una carrera profesional y los que logran egresar de la misma generación, después de acreditar todas las asignaturas correspondientes al currículo de cada carrera, en los tiempos estipulados por los diferentes planes de estudio. Es un indicador que refleja el grado de cumplimiento de los objetivos educativos y la capacidad de retención de los estudiantes. Una baja eficiencia terminal implica una pérdida de recursos humanos, materiales y financieros para el sistema educativo y para la sociedad \cite{fuente2, fuente3}.

Los índices de eficiencia terminal se pueden obtener a partir de los datos de matrícula, egreso y titulación de cada generación de estudiantes. Se pueden expresar en forma de porcentaje o de razón. Por ejemplo, si de 100 alumnos que ingresaron a una carrera en el año 2010, solo 60 egresaron en el año 2014 y 40 se titularon en el año 2016, la eficiencia terminal sería del 60\% y la razón de titulación sería de 0.67. Estos índices se pueden comparar entre diferentes carreras, instituciones, regiones o países, y se pueden analizar en función de variables como el género, el tipo de institución, el tipo de programa, el turno, el nivel socioeconómico, etc. \cite{fuente2, fuente3}.

Las tasas que mencionas son indicadores que se utilizan para evaluar el desempeño y la calidad de la educación superior. A continuación te presento las fórmulas matemáticas para calcular cada una de ellas:
\begin{itemize}
    \item Tasa de deserción: Es el porcentaje de estudiantes que abandonan sus estudios antes de completarlos. Se puede calcular de diferentes formas, pero una de las más comunes es la siguiente \cite{fuente2, fuente3}:
    \[ Tasa\ de\ deserción = \frac{N - G}{N} \times 100 \]
    Donde $N$ es el número de estudiantes que ingresaron al programa y $G$ es el número de estudiantes que se graduaron con un diploma estándar.
    
    \item Tasa de reprobación: Es el porcentaje de estudiantes que no aprueban una asignatura o un curso. Se puede calcular de la siguiente manera:
    \[ Tasa\ de\ reprobación = \frac{R}{M} \times 100 \]
    Donde $R$ es el número de estudiantes que reprobaron la asignatura o el curso y $M$ es el número de estudiantes matriculados en la asignatura o el curso.
    
    \item Tasa de titulación: Es el porcentaje de estudiantes que obtienen el título académico al finalizar sus estudios. Se puede calcular de la siguiente manera:
    \[ Tasa\ de\ titulación = \frac{T}{E} \times 100 \]
    Donde $T$ es el número de estudiantes que se titularon y $E$ es el número de estudiantes egresados del programa.
    
    \item Tasa de egreso: Es el porcentaje de estudiantes que terminan sus estudios y obtienen el certificado de egreso. Se puede calcular de la siguiente manera:
    \[ Tasa\ de\ egreso = \frac{E}{I} \times 100 \]
    Donde $E$ es el número de estudiantes egresados y $I$ es el número de estudiantes inscritos en el programa.
    
    \item Tasa de rezago: Es el porcentaje de estudiantes que no terminan sus estudios en el tiempo establecido. Se puede calcular de la siguiente manera:
    \[ Tasa\ de\ rezago = \frac{Z}{I} \times 100 \]
    Donde $Z$ es el número de estudiantes rezagados y $I$ es el número de estudiantes inscritos en el programa.
\end{itemize}

El rezago educativo es una situación que afecta a muchas personas que no han completado la educación obligatoria o que no asisten a la escuela. El avance académico es el progreso que hacen los estudiantes en sus estudios universitarios, medido por indicadores como la eficacia terminal, la eficacia de egreso, la tasa de titulación, el rendimiento escolar, etc.

Es posible calcular una tasa de rezago educativo o de avance académico en estudios universitarios, pero se requiere de información estadística confiable y actualizada, así como de criterios y metodologías adecuados para hacer las comparaciones y los análisis pertinentes.

Existen algunas fuentes de información que pueden ser útiles para este propósito, como el INEGI, el CONEVAL, el ANUIES, el SEP, entre otras. Estas instituciones publican datos e informes sobre el estado de la educación en México, incluyendo el nivel universitario. Por ejemplo, puedes consultar el siguiente enlace \cite{fuente22} para ver un artículo sobre el rezago educativo en Veracruz, o el siguiente enlace \cite{fuente24} para ver un estudio sobre el contexto académico de estudiantes universitarios en condición de rezago.

Sin embargo, hay que tener en cuenta que los datos disponibles pueden tener limitaciones o variaciones, dependiendo de la fuente, el periodo, la muestra, la definición y el cálculo de los indicadores. Por lo tanto, se recomienda hacer una revisión crítica y una interpretación cuidadosa de la información, así como contrastarla con otras fuentes para obtener una visión más completa y precisa del fenómeno.


\section{reescribiendo}



\begin{thebibliography}{99}
    \bibitem{fuente1} Evaluación formativa: ¿qué es este método ... - Psicología y Mente
    \bibitem{fuente2} Evaluación formativa en el contexto universitario: oportunidades y ...
    \bibitem{fuente3} La evaluación y acreditación de la educación profesional en México: ¿la ...
    \bibitem{fuente4} Evaluación y aprendizaje en educación universitaria - UNAM
    \bibitem{fuente5} La acreditación del aprendizaje por experiencia en la ... - SciELO
    \bibitem{fuente6} Revista de la educación superior - SciELO México
    \bibitem{fuente7} La evaluación formativa en la educación - Redalyc
    \bibitem{fuente8} https://doi.org/10.33595/2226-1478.13.2.672
\bibitem{fuente9} Deserción y Rezago en la Universidad. Indicadores para la Autoevaluación.
    \bibitem{fuente10} Factores asociados al rezago en estudiantes de una institución de ...
    \bibitem{fuente11} Indicadores educativos - INEE
    \bibitem{fuente12} AT02e - 2017 Tasa de eficiencia terminal - INEE
    \bibitem{fuente13} El uso de las TIC como estrategia en la Eficiencia Terminal del ...
    \bibitem{fuente14} Eficiencia terminal en la educación superior, la necesidad de un nuevo ...
    \bibitem{fuente15} INEE | La Educación Obligatoria en México - Informe 2019
    \bibitem{fuente16} Nota técnica sobre el rezago educativo, 2018-2020 - CONEVAL
    \bibitem{fuente17} Cómo calcular la tasa de deserción - Ciencia de Hoy
    \bibitem{fuente18} Cómo calcular la tasa de deserción - Usroasterie.com
    \bibitem{fuente19} Comparación de los índices de deserción, retención, reprobación y ...
    \bibitem{fuente20} Comparación de los índices de deserción, retención, reprobación y ...
    \bibitem{fuente21} http://unesdoc.unesco.org/images/0023/002326/232652S.pdf
 \bibitem{fuente22} Rezago Educativo. Problema sin resolver - Sociedad 3.0
    \bibitem{fuente23} El contexto académico de estudiantes universitarios en condición de ...
    \bibitem{fuente24} Estimaciones del rezago educativo al 31 de diciembre de cada año, INEA ...
    \bibitem{fuente25} Rezago educativo en la infancia y adolescencia de México (2020)
    \bibitem{fuente26} Nota técnica sobre el rezago educativo, 2018-2020 5 de ... - CONEVAL


\end{thebibliography}


\end{document}
